\documentclass[12]{article}

% Packages
\usepackage{listings}
\usepackage[dvipsnames]{xcolor}
\usepackage{amsmath}
\usepackage[left=1.5in, right=1.5in, top=1in, bottom=1in]{geometry}
\usepackage{graphicx}
\usepackage{url}
%\usepackage{pgfplots}
%\usepackage{mathtools}

% Code Block Theme
\lstdefinestyle{mycodestyle}{
	backgroundcolor=\color{white},
	commentstyle=\color{gray},
	keywordstyle=\color{BurntOrange}\bfseries,
	numberstyle=\tiny\color{gray},
	stringstyle=\color{ForestGreen},
	basicstyle=\ttfamily\footnotesize,
	breaklines=true,
	numbers=left,
	frame=single,
	captionpos=b,
	tabsize=4,
	showstringspaces=false
}

% Document
\begin{document}
% Title
\title{Minecraft Parkour}
\author{Jeffrey Hewitt\\ CSCI 004}
\maketitle

% Section 1
\section{Introduction}
Parkour is a diverse exercise that is usually focused on moving from one location to another in the quickest way possible. Performing tricks (freerunning) and focusing on the efficiency of maneuvering around a singular obstacle are some other prominent endeavors in the sport as well. Activities like parkour have been around for centuries at least, although mainly popularized in the last forty years through various forms of media such as television, films, and video games. While some video games have intended parkour movements and courses, being able to move in \textit{any} capacity is all that's needed to prompt users to start finding movement optimizations. Minecraft falls neatly into the latter category as players are able to create and modify structures to form brilliant courses that can be used to test the boundaries of what is possible. This paper will go on to explore how the relationship between physics, collision boxes, and input timing in a tick-based system impacts the creation and difficulty of parkour obstacles in Minecraft.

% Section 2
\section{General Physics}

% Section 3
\section{Vertical Movement}

% Section 4
\section{Horizontal Movement and Momentum}

% Section 5
\section{Collision}

% Section 6
\section{Jumps?}

% Section 7
\section{Math}

% Section 8
\section{Python}

% Section 9
\section{Design?}

% Section 10
\section{Conclusion}

% End of document
\end{document}