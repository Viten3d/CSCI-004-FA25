\documentclass[12]{article}

\usepackage{listings}
\usepackage{xcolor}
\usepackage{amsmath}
%\usepackage{fullpage}
\usepackage[left=1.5in, right=1.5in, top=1in, bottom=1in]{geometry}
\usepackage{graphicx}

\lstdefinestyle{mycodestyle}{
	backgroundcolor=\color{white},
	commentstyle=\color{gray},
	keywordstyle=\color{orange}\bfseries,
	numberstyle=\tiny\color{gray},
	stringstyle=\color{red},
	basicstyle=\ttfamily\footnotesize,
	breaklines=true,
	numbers=left,
	frame=single,
	captionpos=b,
	tabsize=4,
	showstringspaces=false
}

\begin{document}
\title{Pronic Numbers}
\author{Jeffrey Hewitt\\ CSCI 004}
\maketitle

\section{Introduction}

Pronic numbers are formed by the product of two consecutive integers, as shown in Equation \ref{eq:Pronic Formula} below. These numbers can also be thought of as the sum of the first $ n+1\ $even numbers starting at 0, seen in Equation \ref{eq:Pronic Formula Alt}. Some of the earliest references to these numbers date back to the times of Pythagorus and Aristotle, usually denoted as \textit{oblong} or \textit{rectangular} in relation to \textit{figurate} numbers. Equation \ref{eq:Triangular Formula} shows that pronic numbers are just two times the value of the more commonly known figurates: \textit{triangular} numbers.

\begin{equation}
	P_{n} = n(n+1) \label{eq:Pronic Formula}
\end{equation}

\begin{equation}
	P_{n} = \sum_{k = 0}^{n}2k \label{eq:Pronic Formula Alt}
\end{equation}

\begin{equation}
	P_{3} = 3(3 + 1) = 0 + 2 + 4 + 6 = 12 \label{eq:Pronic Formula Example}
\end{equation}

\begin{equation}
	T_{n} = \frac{n(n+1)}{2} = \sum_{k = 0}^{n}k = \frac{P_{n}}{2} \label{eq:Triangular Formula}
\end{equation}

\section{Figurate Geometry} \label{sec:Figurate Geometry}

Expressing figurate numbers through geometric illustrations can be helpful for understanding \textit{why} the formulas are what they are. Pronic numbers in particular can be mapped to a rectangle with side lengths $ n\ $and $ n+1\ $, demonstrated in Figure \ref{fig:Pronic Rectangles}. The area of these rectangles would be pronic numbers: $ A = L \times W = (n+1) \times n = P_{n} \ $. Using dots to represent these polygons can assist in proving that pronics\footnote{``Pronics" may occasionally be used in place of ``pronic numbers" to avoid word repetition.} are in fact the sum of consecutive, even integers by grouping \textit{gnomons} — smaller, similar polygons within the main shape — as indicated by the red dots in Figure \ref{fig:Pronic Rectangles1}.

\begin{figure}[h]
	\centering
	\includegraphics[width=80mm]{C:/Users/Jeff/Documents/college/CSCI4/Tex/pronics_paper/pronic_shapes.PNG}
	\caption{Pronic Rectangles}
	\label{fig:Pronic Rectangles}
\end{figure}

\begin{figure}[h]
	\centering
	\includegraphics[width=80mm]{C:/Users/Jeff/Documents/college/CSCI4/Tex/pronics_paper/pronic_shapes1.PNG}
	\caption{Pronic Gnomons}
	\label{fig:Pronic Rectangles1}
\end{figure}

\section{Pronics in Python}

The code needed to generate individual pronic numbers is fairly straightforward using Equation \ref{eq:Pronic Formula} in a function:

\begin{lstlisting}[style=mycodestyle, language=Python]
def pronic(n):                           #name function and set argument
	return n * (n + 1)                   #output formula result
\end{lstlisting}

Creating a sequence of $ n\ $pronic numbers just requires a basic \textit{for} loop to add the results from Equation \ref{eq:Pronic Formula} to a list, as such:

% , caption={Funciton for listing $ n\ $pronic numbers.}
\begin{lstlisting}[style=mycodestyle, language=Python]
def pronics(k):                          #name function and set argument
	pronics_list = []                    #make empty list to store values
	for n in range(0, k):                #iterate formula `k' times
		pronics_list.append(n * (n + 1)) #add formula results to list
	return pronics_list                  #output entire list
\end{lstlisting}

Or, to generate the sequence up to $ P_{n}\ $:

\begin{lstlisting}[style=mycodestyle, language=Python]
def pronics_alt(k):                   
	pronics_alt_list = []                    
	for n in range(0, k + 1):            #iterate formula `k + 1' times
		pronics_alt_list.append(n * (n + 1)) 
	return pronics_alt_list                  
\end{lstlisting}

Each with their respective outputs here:

\begin{lstlisting}[style=mycodestyle, language=Python]
print(pronic(5))
Output: 30
	
print(pronics(10)) 
Output: [0, 2, 6, 12, 20, 30, 42, 56, 72, 90]
	
print(pronics_alt(10))
Output: [0, 2, 6, 12, 20, 30, 42, 56, 72, 90, 110]
\end{lstlisting}

\section{Pronic Properties}

There are various methods of transforming any sequence into new and interesting ``structures", whether it be just another sequence or the convergence of a series to some constant. One such transformation can be seen in the following proof which uses limits to establish that the series of pronic reciprocals converges to `1'.

\begin{align*}
	\sum_{n = 1}^{\infty}\frac{1}{n(n+1)} &= \lim_{k\to\infty}\sum_{n = 1}^{k}\frac{1}{n(n+1)} \\
	&= \lim_{k\to\infty}\sum_{n = 1}^{k}\left(\frac{1}{n} - \frac{1}{n+1}\right) \\
	&= \lim_{k\to\infty}\left[\left(\frac{1}{1} - \frac{1}{2}\right) + \left(\frac{1}{2} - \frac{1}{3}\right) + \left(\frac{1}{3} - \frac{1}{4}\right) + \dots + \left(\frac{1}{k} - \frac{1}{k+1}\right)\right] \\
	&= \lim_{k\to\infty}\left[1 + \left(-\frac{1}{2} + \frac{1}{2}\right) + \left(-\frac{1}{3} + \frac{1}{3}\right) + \dots + \left(-\frac{1}{k} + \frac{1}{k}\right) -\frac{1}{k+1}\right] \\
	&= \lim_{k\to\infty}\left[1 - \frac{1}{k+1}\right] = 1
\end{align*}

This can be observed in python with the following program:

\begin{lstlisting}[style=mycodestyle, language=Python]
k = 10000000                             #upper limit from prior proof
svar = 0                                 #temp var for summation

for n in range(1, k):
	fvar = 1 / (n * (n + 1))             #calculate pronic reciprocal
	svar = svar + fvar                   #add terms together
	if n % (k / 10) == 0 or n <= 10:     #optional statement to output
		print(svar , n)                    #values as program runs
\end{lstlisting}

\begin{lstlisting}[style=mycodestyle, language=Python]
print(svar)
Output: 0.9999998999998053
\end{lstlisting}

Personally, I find it more intriguing when series converge to irrational constants represented by manipulations of other constants like $ e\ $and $ \pi\,$, or by logarithms and \textit{nth} roots. The next transformation follows that trend of \textit{irrational convergence} by taking the alternating sum of the pronic reciprocals using $ (-1)^{n+1}\ $ to switch signs between terms.

\begin{equation*}
	\sum_{n = 1}^{\infty}\frac{(-1)^{n+1}}{n(n+1)} = \frac{1}{2} - \frac{1}{6} + \frac{1}{12} - \frac{1}{20} + \dots = \ln(4) - 1
\end{equation*}

The result from this series can be demonstrated using a program to iterate the sum and compare that with the value of $ \log(4) - 1\,$:

\begin{lstlisting}[style=mycodestyle, language=Python]
from math import log                    #import log function from math module
	
k = 10000000                            #upper limit
svar = 0                                #temp var for summation
	
for n in range(1, k):
	fvar = ((-1) ** (n + 1)) / (n * (n + 1)) #calculate alternating sum
	svar = svar + fvar                  #add terms together
\end{lstlisting}

\begin{lstlisting}[style=mycodestyle, language=Python]
	print(svar)
	Output: 0.38629436111989124
	
	print(log(4) - 1)
	Output: 0.3862943611198906
\end{lstlisting}

\section*{References}

\end{document}